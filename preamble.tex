%パッケージ導入

\usepackage{amsmath,amssymb,ascmac,fancybox}	%基本パッケージ
\usepackage[dvipdfmx]{color}
\usepackage[dvipdfmx]{graphicx}
\usepackage{url}
\usepackage{siunitx}
\usepackage{layout}		%layout確認用
\usepackage{tcolorbox}		%tcolorboxとその拡張用パッケージ
	\tcbuselibrary{raster,skins}	
	\tcbuselibrary{xparse}
	\tcbuselibrary{breakable}
\usepackage{fancyhdr}		%ページスタイル用パッケージ
\usepackage{subcaption}		%並べた図に異なる見出しを付ける為のパッケージ
\usepackage{mhchem}		%化学式など用パッケージ
\usepackage{bm}
\usepackage{physics}
%パッケージ導入終わり

%graphicフォルダ指定
\graphicspath{{graphics/}}

%各種マクロ設定
\newcommand{\emp}[1]{\textcolor{red}{\textgt{\bf #1}}}	%太字、赤色
\newcommand{\gtb}[1]{\textgt{\textbf{#1}}}	%太字ゴシック
\newcommand{\ex}{\underline{\gtb{例}}}		%「例」の字 太字と下線付き
\newcommand{\refer}[1]{(\ref{#1})}		%括弧つき式番号等参照
\newcommand{\partdif}[2]{\frac{\partial #1}{\partial #2}}	%偏微分 \pardif{a}{b}でaをbで偏微分した表記になる
\newcommand{\substi}[2]{\left.#1\right|_{#2}}		%縦線代入
\newcommand{\Vector}[3]{\left( \begin{array}{c} #1 \\ #2 \\ #3 \end{array} \right)}	%ベクトル成分表示


%各種環境
\newenvironment{column}[1]{\begin{boxnote}{\large \textbf{\textgt{コラム #1}}} \vspace{3pt} \\}{\end{boxnote}}	%コラムボックス環境
\newenvironment{matome}{\begin{tcolorbox}[enhanced,title= \gtb{まとめ},		
  	attach boxed title to top left={xshift=3mm, yshift*=-\tcboxedtitleheight/2}]}
{\end{tcolorbox}}												%まとめボックス環境